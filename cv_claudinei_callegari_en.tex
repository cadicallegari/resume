%%%%%%%%%%%%%%%%%%%%%%%%%%%%%%%%%%%%%%%%%
% Medium Length Professional CV
% LaTeX Template
% Version 3.0 (December 17, 2022)
%
% This template originates from:
% https://www.LaTeXTemplates.com
%
% Author:
% Vel (vel@latextemplates.com)
%
% Original author:
% Trey Hunner (http://www.treyhunner.com/)
%
% License:
% CC BY-NC-SA 4.0 (https://creativecommons.org/licenses/by-nc-sa/4.0/)
%
%%%%%%%%%%%%%%%%%%%%%%%%%%%%%%%%%%%%%%%%%

\title{Callegari Resume}
\author{Claudinei Callegari}

%----------------------------------------------------------------------------------------
%	PACKAGES AND OTHER DOCUMENT CONFIGURATIONS
%----------------------------------------------------------------------------------------

\documentclass[
	a4paper, % Uncomment for A4 paper size (default is US letter)
	10pt, % Default font size, can use 10pt, 11pt or 12pt
]{resume} % Use the resume class

\usepackage{fontenc}

\setlength{\textheight}{10in} % increase text height to fit resume on 1 page


%------------------------------------------------

\name{Claudinei Callegari} % Your name to appear at the top

% You can use the \address command up to 3 times for 3 different addresses or pieces of contact information
% Any new lines (\\) you use in the \address commands will be converted to symbols, so each address will appear as a single line.

\address{Wuppertal, Germany} % Main address

\address{cadicallegari@gmail.com \\ +49 152 3663 8091} % Contact information

\address{\textbf{Github}: github.com/cadicallegari \\ \textbf{Linkedin}: linkedin.com/in/cadicallegari} % Contact information

%----------------------------------------------------------------------------------------

\begin{document}


%----------------------------------------------------------------------------------------
%	INTRO SECTION
%----------------------------------------------------------------------------------------

\begin{rSection}{Summary}

I've been a software engineer solving product, scalability, and developer problems with Go and microservices since 2016.

I'm a big fan of automated tests and fast feedback loops, and I believe that collaborative and relaxed work environments walk together with productivity and creativity.

Apart from Go, I have worked with some other languages like C, Python, and Ruby.
I'm also interested in the functional paradigm, and I have played around using Scala and Clojure.

\end{rSection}

%----------------------------------------------------------------------------------------
%	EDUCATION SECTION
%----------------------------------------------------------------------------------------

\begin{rSection}{Education}

	\textbf{Universidade Estadual do Oeste do Paraná, Brazil} \hfill \textbf{December 2011} \\
	B.S. in Computer Science

\end{rSection}


%----------------------------------------------------------------------------------------
%	WORK EXPERIENCE SECTION
%----------------------------------------------------------------------------------------

\begin{rSection}{Experience}

	\begin{rSubsection}{Schwarz Media Plataform}{April 2022 - Present}{Senior Software Engineer}{Germany}
		\item[]
		As a member of the ad delivery team, some of my responsibilities were to keep our service within our SLOs, keep the pipeline that prepared the ads to be delivered,
		and onboard and share knowledge with other team members.

		The team had a rotative style of dealing with new product initiatives. For every initiative, a different team member was responsible for the design and implementation of the solution.
		writing tech concepts, ADRs, and coordinating the work within the team as well as with other teams in the company.

		\textbf{Highlights}
		\begin{itemize}
			\item Design bigtable storage to make it simpler from a developer perspective and also more performant.
			\item Improve overall developer experience with a local environment.
			\item Migrate a complex pipeline from kubeflow to airflow.
		\end{itemize}

		\textbf{Main technologies}
			Go, Bigtable, Bigquery, Airflow, Kubernetes, Docker, GCP, Postgres.

	\end{rSubsection}

	\vspace{1mm}
%------------------------------------------------

	\begin{rSubsection}{Onefootball}{February 2021 - March 2022}{Senior Software Engineer}{Berlin, Germany}
		\item[]
		As a member of the news team, I share the responsibility to keep the data provided for our users updated and tailored for their interests.
		Among other interesting challenges, scaling our services to support a high amount of requests per second, with spikes that can multiply the number of requests in an order of magnitude within a few seconds.

		\textbf{Highlights}
		\begin{itemize}
			\item Responsible for the BFF and event architecture initiatives, which were adopted successfully by other teams later.
			\item Fine tune relational database that improved performance up to 70\%.
		\end{itemize}

		\textbf{Main technologies}
			Go, Kubernetes, Docker, AWS, Elastic Search, Nats, RabbitMQ, MySql.

	\end{rSubsection}

	\vspace{1mm}
%------------------------------------------------

	\begin{rSubsection}{FromAtoB}{May 2019 - February 2021}{Senior Software Engineer}{Berlin, Germany}
		\item[]
		As a member of the data platform team, some of my responsibilities were to externalize, process data from different kinds of services written in Go and Ruby,
		and store it in our dataware house and datalake.
		Also monitoring to make sure that the process still working and generating quality data.
		With that, we enabled the data science teams to work with the data to improve our product as well as create new features like price calendars, for example.

		\textbf{Highlights}
		\begin{itemize}
			\item Design and implement the whole data platform.
			\item Create a POC of a price calendar using Bigtable.
		\end{itemize}

		\textbf{Main technologies}
			Go, Bigtable, Bigquery, PubSub, Dataflow, Cloud Functions, Airflow, Kubernetes, Docker, GCP.


	\end{rSubsection}

	\vspace{1mm}
%------------------------------------------------

	\begin{rSubsection}{Neoway Business Solutions}{May 2016 - March 2019}{Software Developer}{Florianópolis, Brazil}
		\item[]
		As a member of the data source, our main responsibility was to keep a fleet of more than 300 bots scraping different sites on the Internet.
		For that, we created a framework over Scrapy to help us develop bots in a more rapid and reliable way.
		We also had to build a variety of services in Go to support the bots with storage and proxy, for example.

		Later, we applied the same framework idea to the data analytics team, which helped to develop their models in a more quickly and reliable way.
		having as the final artifact a variety of options, like a Python package or a Spark job.

		\textbf{Highlights}
		\begin{itemize}
			\item Promote workshops and regular 1x1 with the team.
			\item Improve and replicate with another team the framework idea to make the development process easier.
		\end{itemize}

		\textbf{Main technologies}
			Go, Python, RabbitMQ, Kubernetes, Docker, AWS, GCP.

	\end{rSubsection}

	%------------------------------------------------

	\begin{rSubsection}{DBA Technology}{May 2013 - May 2016}{Software Developer}{Florianópolis, Brazil}
		\item[]
		As a team member of the DBA startup, I've worked on the Parking Meter System project, where I was involved in all stages, like collecting and understanding the requirements, web development using the Ruby on Rails framework, modeling the database, and testing the application.

		Apart from the frontend, we had to communicate with other devices, such as tablets, parking meters, and a variety of sensors. That API was developed using the Ruby/Grape framework.

		I also wrote the first version of the Android application using libraries like RxAndroid, Retrofit, SqlBrite, Butterniffe, Dagger, Espresso, and Robolectric.

		\textbf{Highlights}
		\begin{itemize}
			\item Start the system from scratch, being responsible for collecting requirements, implementing them, and delivering them to the users..
		\end{itemize}

		\textbf{Main technologies}
			Ruby, Python, Android, AWS.

	\end{rSubsection}

\end{rSection}


%----------------------------------------------------------------------------------------
%	TECHNICAL STRENGTHS SECTION
%----------------------------------------------------------------------------------------

% \begin{rSection}{Technical Strengths}

% 	\begin{tabular}{@{} >{\bfseries}l @{\hspace{6ex}} l @{}}
% 		% Computer Languages & Go, Python, Ruby, C \\
% 		& Go, Python, Ruby, C \\
% 		& Docker, Kubernetes, Microservices \\
% 		& GRPC, REST \\
% 		& AWS, GCP, BigTable, Bigquery, Dataflow \\
% 		& Linux, Postgres, MySQL, Airflow, Android
% 	\end{tabular}

% \end{rSection}

%----------------------------------------------------------------------------------------
%	EXAMPLE SECTION
%----------------------------------------------------------------------------------------

%\begin{rSection}{Section Name}

	%Section content\ldots

%\end{rSection}

%----------------------------------------------------------------------------------------

\end{document}
